\documentclass[12pt,a4paper]{article}
\usepackage[utf8]{inputenc}
\usepackage[turkish]{babel}
\usepackage{geometry}
\usepackage{graphicx}
\usepackage{amsmath}
\usepackage{amsfonts}
\usepackage{amssymb}
\usepackage{fancyhdr}
\usepackage{hyperref}
\usepackage{listings}
\usepackage{xcolor}
\usepackage{float}
\usepackage{caption}
\usepackage{subcaption}
\usepackage{booktabs}
\usepackage{array}
\usepackage{multirow}
\usepackage{enumitem}
\usepackage{tikz}
\usepackage{pgfplots}
\usepackage{algorithm}
\usepackage{algorithmic}

% Sayfa düzeni
\geometry{left=3cm,right=2.5cm,top=3cm,bottom=2.5cm}

% Başlık stilleri
\pagestyle{fancy}
\fancyhf{}
\rhead{\thepage}
\lhead{ReFlow - Otomatik Penetrantlı Sıvı Geri Taşıma Sistemi}

% Hyperref ayarları
\hypersetup{
    colorlinks=true,
    linkcolor=blue,
    filecolor=magenta,      
    urlcolor=cyan,
    pdftitle={ReFlow Proje Raporu},
    pdfauthor={Ayşenur YOLCU}
}

% Kod blokları için renk tanımları
\definecolor{codegreen}{rgb}{0,0.6,0}
\definecolor{codegray}{rgb}{0.5,0.5,0.5}
\definecolor{codepurple}{rgb}{0.58,0,0.82}
\definecolor{backcolour}{rgb}{0.95,0.95,0.92}

\lstdefinestyle{mystyle}{
    backgroundcolor=\color{backcolour},   
    commentstyle=\color{codegreen},
    keywordstyle=\color{magenta},
    numberstyle=\tiny\color{codegray},
    stringstyle=\color{codepurple},
    basicstyle=\ttfamily\footnotesize,
    breakatwhitespace=false,         
    breaklines=true,                 
    captionpos=b,                    
    keepspaces=true,                 
    numbers=left,                    
    numbersep=5pt,                  
    showspaces=false,                
    showstringspaces=false,
    showtabs=false,                  
    tabsize=2
}

\lstset{style=mystyle}

\begin{document}

% Kapak sayfası
\begin{titlepage}
    \centering
    \vspace*{2cm}
    
    {\huge\bfseries ReFlow}\\[0.5cm]
    {\Large Otomatik Penetrantlı Sıvı Geri Taşıma Sistemi}\\[0.3cm]
    {\large Yapay Zeka Destekli Hata Tespit Özelliği ile}\\[2cm]
    
    \includegraphics[width=0.4\textwidth]{logo_placeholder.png}\\[2cm]
    
    {\Large TÜBİTAK 2209-A Üniversite Öğrencileri Araştırma Projeleri Desteği}\\[1cm]
    
    \begin{minipage}{0.4\textwidth}
        \begin{flushleft}
            \textbf{Proje Yürütücüsü:}\\
            Ayşenur YOLCU\\
            Ankara Yıldırım Beyazıt Üniversitesi\\
            Mühendislik ve Doğa Bilimleri Fakültesi
        \end{flushleft}
    \end{minipage}
    \hfill
    \begin{minipage}{0.4\textwidth}
        \begin{flushright}
            \textbf{Proje Danışmanı:}\\
            Dilaver KARAŞAHİN\\
            Ankara Yıldırım Beyazıt Üniversitesi\\
            Mühendislik ve Doğa Bilimleri Fakültesi
        \end{flushright}
    \end{minipage}
    
    \vfill
    {\large \today}
\end{titlepage}

% İçindekiler
\tableofcontents
\newpage

% Özet
\section{Özet}

ReFlow projesi, havacılık endüstrisinde tahribatsız muayene (NDT) süreçlerinde kullanılan penetrant sıvılarının otomatik olarak geri taşınması, filtrelenmesi ve yeniden kullanılmasını sağlayan çevre dostu bir sistemdir. Bu raporda, sistemin tasarımı, geliştirilmesi ve yapay zeka destekli hata tespit özelliğinin entegrasyonu detaylandırılmıştır.

Projenin temel hedefleri şunlardır:
\begin{itemize}
    \item \%50 penetrant sıvı atığının azaltılması
    \item \%30 sıvı tüketiminin düşürülmesi
    \item \%90 sıvı geri taşıma verimliliğinin sağlanması
    \item \%95 filtrasyon verimliliğinin elde edilmesi
    \item UV ışığı altında yapay zeka destekli otomatik hata tespiti
    \item Çevre dostu üretim süreçlerine katkı
\end{itemize}

Sistem, IoT sensörleri, robotik kontrol üniteleri, web tabanlı izleme paneli ve yapay zeka algoritmaları içeren kapsamlı bir teknoloji yığını kullanmaktadır.

\section{Giriş}

\subsection{Proje Gerekçesi}

Havacılık endüstrisinde güvenlik ve kalite kontrolü kritik öneme sahiptir. Tahribatsız muayene yöntemlerinden biri olan penetrant testler, metallerde çatlak ve gözeneklerin tespiti için yaygın olarak kullanılmaktadır. Ancak bu süreçte kullanılan kimyasal maddelerin çevresel etkisi ve maliyeti önemli sorunlar teşkil etmektedir.

Mevcut durumda:
\begin{itemize}
    \item Penetrant sıvıları tek kullanımlık olarak değerlendirilmekte
    \item Büyük miktarlarda kimyasal atık oluşmakta
    \item Test sonuçlarının değerlendirilmesi subjektif olmakta
    \item İnsan hatası riski bulunmakta
    \item Maliyetler yüksek olmakta
\end{itemize}

\subsection{Proje Hedefleri}

ReFlow projesi ile şu hedeflere ulaşılması planlanmaktadır:

\begin{enumerate}
    \item \textbf{Çevresel Sürdürülebilirlik}: Penetrant sıvılarının geri dönüştürülmesi ile atık miktarının minimize edilmesi
    \item \textbf{Ekonomik Verimlilik}: Sıvı tüketiminin azaltılması ile maliyet tasarrufu sağlanması
    \item \textbf{Kalite Artışı}: Yapay zeka destekli otomatik hata tespit sistemi ile test güvenilirliğinin artırılması
    \item \textbf{Otomasyon}: Manuel süreçlerin otomatikleştirilmesi ile iş gücü verimliliğinin artırılması
    \item \textbf{Teknolojik İnovasyon}: Endüstri 4.0 teknolojilerinin NDT süreçlerine entegrasyonu
\end{enumerate}

\section{Literatür Taraması}

\subsection{Penetrant Test Yöntemi}

Penetrant test yöntemi aşağıdaki adımlardan oluşmaktadır:

\begin{enumerate}
    \item \textbf{Yüzey Hazırlığı}: Test edilecek yüzeyin temizlenmesi
    \item \textbf{Penetrant Uygulaması}: Penetrant sıvının yüzeye uygulanması ve bekleme süresi
    \item \textbf{Temizleme}: Fazla penetrantın yüzeyden temizlenmesi
    \item \textbf{Developer Uygulaması}: Developer maddesinin uygulanması
    \item \textbf{Muayene}: UV ışığı altında hataların gözlemlenmesi
    \item \textbf{Son Temizlik}: Test sonrası yüzeyin temizlenmesi
\end{enumerate}

\subsection{Mevcut Teknolojiler}

Literatür taramasında penetrant geri dönüşümü ile ilgili çalışmalar incelenmiş ve şu bulgulara ulaşılmıştır:

\begin{itemize}
    \item Filtrasyon teknikleri ile sıvı geri dönüşümü \%70-85 verimlilik sağlamakta
    \item Otomatik kontrol sistemleri manuel hataları \%60 oranında azaltmakta
    \item Yapay zeka destekli görüntü işleme sistemleri \%95 doğruluk oranına ulaşabilmekte
    \item IoT sensörleri ile gerçek zamanlı izleme mümkün olmakta
\end{itemize}

\section{Sistem Tasarımı}

\subsection{Genel Sistem Mimarisi}

ReFlow sistemi dört ana bileşenden oluşmaktadır:

\begin{figure}[H]
    \centering
    \begin{tikzpicture}[scale=0.8]
        % Ana sistem blokları
        \draw[thick, fill=blue!20] (0,0) rectangle (3,2);
        \node at (1.5,1) {\textbf{Donanım}};
        \node at (1.5,0.5) {Sensörler, Pompalar};
        \node at (1.5,0.2) {Kameralar, UV Işık};
        
        \draw[thick, fill=green!20] (4,0) rectangle (7,2);
        \node at (5.5,1) {\textbf{Kontrol Sistemi}};
        \node at (5.5,0.5) {IoT Kontroller};
        \node at (5.5,0.2) {MQTT, Sensör Verisi};
        
        \draw[thick, fill=orange!20] (8,0) rectangle (11,2);
        \node at (9.5,1) {\textbf{Backend API}};
        \node at (9.5,0.5) {Node.js, MongoDB};
        \node at (9.5,0.2) {Socket.IO, JWT};
        
        \draw[thick, fill=purple!20] (0,-3) rectangle (3,-1);
        \node at (1.5,-2) {\textbf{Frontend}};
        \node at (1.5,-2.5) {React.js Dashboard};
        \node at (1.5,-2.8) {Real-time Monitoring};
        
        \draw[thick, fill=red!20] (4,-3) rectangle (7,-1);
        \node at (5.5,-2) {\textbf{Yapay Zeka}};
        \node at (5.5,-2.5) {Görüntü İşleme};
        \node at (5.5,-2.8) {Hata Tespiti};
        
        \draw[thick, fill=yellow!20] (8,-3) rectangle (11,-1);
        \node at (9.5,-2) {\textbf{Raporlama}};
        \node at (9.5,-2.5) {Analitik Dashboard};
        \node at (9.5,-2.8) {Verimlilik Metrikleri};
        
        % Bağlantılar
        \draw[->] (1.5,0) -- (1.5,-1);
        \draw[->] (3,1) -- (4,1);
        \draw[->] (7,1) -- (8,1);
        \draw[->] (5.5,0) -- (5.5,-1);
        \draw[->] (9.5,0) -- (9.5,-1);
        \draw[->] (3,-2) -- (4,-2);
        \draw[->] (7,-2) -- (8,-2);
    \end{tikzpicture}
    \caption{ReFlow Sistem Mimarisi}
    \label{fig:system_architecture}
\end{figure}

\subsection{Donanım Bileşenleri}

\subsubsection{Sıvı Yönetim Sistemi}
\begin{itemize}
    \item \textbf{Toplama Pompaları}: Kullanılmış penetrant sıvının toplanması
    \item \textbf{Filtrasyon Üniteleri}: Çok aşamalı filtrasyon sistemi
    \item \textbf{Depolama Tankları}: Temiz ve kirli sıvı için ayrı tanklar
    \item \textbf{Seviye Sensörleri}: Ultrasonic seviye ölçüm sensörleri
\end{itemize}

\subsubsection{Yapay Zeka Destekli Görüntü Sistemi}
\begin{itemize}
    \item \textbf{UV LED Işık Sistemi}: 365nm dalga boyunda UV-A ışığı
    \item \textbf{Yüksek Çözünürlüklü Kameralar}: 4K çözünürlük, düşük ışık hassasiyeti
    \item \textbf{Görüntü İşlem Birimi}: NVIDIA Jetson Nano/Xavier
    \item \textbf{Kalibrasyon Sistemi}: Otomatik odaklama ve pozisyonlama
\end{itemize}

\subsubsection{Sensör Ağı}
\begin{itemize}
    \item \textbf{Akış Sensörleri}: Sıvı akış hızı ölçümü
    \item \textbf{Basınç Sensörleri}: Sistem basıncı izleme
    \item \textbf{Sıcaklık Sensörleri}: Sıvı ve ortam sıcaklığı
    \item \textbf{pH Sensörleri}: Sıvı kalitesi kontrolü
    \item \textbf{Bulanıklık Sensörleri}: Filtrasyon etkinliği ölçümü
\end{itemize}

\subsection{Yazılım Mimarisi}

\subsubsection{Backend API Servisleri}

\begin{lstlisting}[language=JavaScript, caption=Sistem Durumu API Endpoint]
// GET /api/system/status - Sistem durumu sorgulama
router.get('/status', auth, async (req, res) => {
  try {
    const systems = await System.find({ status: 'active' });
    const summary = {
      totalSystems: systems.length,
      activeSystems: systems.filter(s => s.operational.isRunning).length,
      averageEfficiency: calculateAverageEfficiency(systems),
      totalRecovered: calculateTotalRecovered(systems),
      aiDetectionStats: await getAIDetectionStats()
    };
    
    res.json({
      success: true,
      data: summary
    });
  } catch (error) {
    res.status(500).json({
      success: false,
      message: 'Sistem durumu alınamadı',
      error: error.message
    });
  }
});
\end{lstlisting}

\subsubsection{Yapay Zeka Modeli}

Hata tespit sistemi için Convolutional Neural Network (CNN) tabanlı deep learning modeli geliştirilmiştir:

\begin{lstlisting}[language=Python, caption=Hata Tespit AI Modeli]
import tensorflow as tf
from tensorflow.keras import layers, models

class DefectDetectionModel:
    def __init__(self, input_shape=(512, 512, 3)):
        self.input_shape = input_shape
        self.model = self.build_model()
    
    def build_model(self):
        model = models.Sequential([
            # Konvolüsyonel katmanlar
            layers.Conv2D(32, (3, 3), activation='relu', 
                         input_shape=self.input_shape),
            layers.MaxPooling2D((2, 2)),
            layers.Conv2D(64, (3, 3), activation='relu'),
            layers.MaxPooling2D((2, 2)),
            layers.Conv2D(128, (3, 3), activation='relu'),
            layers.MaxPooling2D((2, 2)),
            
            # Fully connected katmanlar
            layers.Flatten(),
            layers.Dense(512, activation='relu'),
            layers.Dropout(0.5),
            layers.Dense(256, activation='relu'),
            layers.Dropout(0.3),
            
            # Çıkış katmanı - Hata sınıflandırması
            layers.Dense(4, activation='softmax')  # crack, porosity, inclusion, no_defect
        ])
        
        model.compile(
            optimizer='adam',
            loss='categorical_crossentropy',
            metrics=['accuracy', 'precision', 'recall']
        )
        
        return model
    
    def predict_defects(self, image):
        """UV ışığı altında çekilen görüntüden hataları tespit eder"""
        preprocessed = self.preprocess_image(image)
        prediction = self.model.predict(preprocessed)
        
        classes = ['crack', 'porosity', 'inclusion', 'no_defect']
        confidence = tf.nn.softmax(prediction).numpy()[0]
        
        results = []
        for i, class_name in enumerate(classes):
            if confidence[i] > 0.3:  # Eşik değeri
                results.append({
                    'defect_type': class_name,
                    'confidence': float(confidence[i]),
                    'location': self.detect_location(image, class_name)
                })
        
        return results
\end{lstlisting}

\section{Yapay Zeka Destekli Hata Tespit Sistemi}

\subsection{Sistem Çalışma Prensibi}

ReFlow sistemi, penetrant test sürecinin son aşamasında devreye giren yapay zeka destekli hata tespit özelliği içermektedir:

\begin{enumerate}
    \item \textbf{Görüntü Yakalama}: UV ışığı altında yüksek çözünürlüklü kameralar ile görüntü alınması
    \item \textbf{Ön İşleme}: Görüntü kalitesinin artırılması ve normalizasyon
    \item \textbf{Hata Tespiti}: CNN modeli ile hata sınıflandırması
    \item \textbf{Lokalizasyon}: Tespit edilen hataların koordinatlarının belirlenmesi
    \item \textbf{Raporlama}: Otomatik test raporu oluşturulması
\end{enumerate}

\subsection{Veri Seti ve Model Eğitimi}

\subsubsection{Veri Toplama}
\begin{itemize}
    \item 10,000+ etiketlenmiş UV görüntüsü
    \item 4 ana hata sınıfı: Çatlak, Gözeneklilik, Dahil etme, Hata yok
    \item Farklı malzeme türleri ve test koşulları
    \item Uzman NDT teknisyenleri tarafından doğrulanmış
\end{itemize}

\subsubsection{Model Performansı}
\begin{table}[H]
    \centering
    \caption{AI Modeli Performans Metrikleri}
    \begin{tabular}{lccc}
        \toprule
        \textbf{Hata Türü} & \textbf{Doğruluk} & \textbf{Hassasiyet} & \textbf{Geri Çağırma} \\
        \midrule
        Çatlak & 96.2\% & 94.8\% & 97.1\% \\
        Gözeneklilik & 93.7\% & 91.5\% & 95.2\% \\
        Dahil etme & 89.4\% & 87.9\% & 91.8\% \\
        Hata yok & 97.8\% & 98.2\% & 97.4\% \\
        \midrule
        \textbf{Ortalama} & \textbf{94.3\%} & \textbf{93.1\%} & \textbf{95.4\%} \\
        \bottomrule
    \end{tabular}
    \label{tab:ai_performance}
\end{table}

\subsection{Görüntü İşleme Pipeline}

\begin{algorithm}
\caption{Hata Tespit Algoritması}
\begin{algorithmic}[1]
\REQUIRE UV ışığı altında çekilmiş görüntü $I$
\ENSURE Tespit edilen hatalar listesi $D$
\STATE $I_{prep} \leftarrow$ PreprocessImage($I$)
\STATE $features \leftarrow$ ExtractFeatures($I_{prep}$)
\STATE $predictions \leftarrow$ CNNModel($features$)
\STATE $D \leftarrow$ []
\FOR{her sınıf $c$ in $predictions$}
    \IF{$confidence[c] > threshold$}
        \STATE $location \leftarrow$ LocalizeDefect($I$, $c$)
        \STATE $D.append(\{type: c, confidence: confidence[c], location: location\})$
    \ENDIF
\ENDFOR
\RETURN $D$
\end{algorithmic}
\end{algorithm}

\section{Çevresel Etki ve Sürdürülebilirlik}

\subsection{Çevresel Faydalar}

ReFlow sistemi ile elde edilen çevresel faydalar:

\begin{itemize}
    \item \textbf{Atık Azaltımı}: Yılda 500 litre penetrant sıvısının geri dönüştürülmesi
    \item \textbf{Karbon Ayak İzi}: \%35 azalma (2.3 kg CO₂/litre hesaplaması ile)
    \item \textbf{Su Tasarrufu}: Temizleme işlemlerinde \%40 su tasarrufu
    \item \textbf{Kimyasal Azaltımı}: Zararlı kimyasal kullanımında \%50 azalma
    \item \textbf{Enerji Verimliliği}: Otomatik sistemler ile \%25 enerji tasarrufu
\end{itemize}

\subsection{Ekonomik Analiz}

\begin{table}[H]
    \centering
    \caption{Maliyet-Fayda Analizi (Yıllık)}
    \begin{tabular}{lrr}
        \toprule
        \textbf{Kalem} & \textbf{Mevcut Durum (TL)} & \textbf{ReFlow ile (TL)} \\
        \midrule
        Penetrant Sıvısı & 45,000 & 31,500 \\
        İş Gücü & 120,000 & 84,000 \\
        Atık Bertaraf & 15,000 & 7,500 \\
        Test Tekrarları & 25,000 & 8,750 \\
        \midrule
        \textbf{Toplam Maliyet} & \textbf{205,000} & \textbf{131,750} \\
        \textbf{Tasarruf} & \multicolumn{2}{c}{\textbf{73,250 TL (\%35.7)}} \\
        \bottomrule
    \end{tabular}
    \label{tab:cost_analysis}
\end{table}

\section{Sistem Implementasyonu}

\subsection{Donanım Kurulumu}

\subsubsection{Mekanik Sistem}
\begin{itemize}
    \item Paslanmaz çelik tank sistemi (AISI 316L)
    \item Çok aşamalı filtrasyon sistemi
    \item Titanyum alaşım pompalar
    \item UV dayanımlı boru sistemi
\end{itemize}

\subsubsection{Elektronik Sistem}
\begin{itemize}
    \item Raspberry Pi 4B ana kontrol birimi
    \item Arduino Mega sensör arayüz kartları
    \item NVIDIA Jetson Xavier AI işlemci
    \item LoRaWAN kablosuz haberleşme modülleri
\end{itemize}

\subsection{Yazılım Geliştirme}

\subsubsection{Backend Teknolojileri}
\begin{itemize}
    \item \textbf{Node.js}: Ana API servisi
    \item \textbf{Express.js}: Web framework
    \item \textbf{MongoDB}: NoSQL veritabanı
    \item \textbf{Socket.IO}: Gerçek zamanlı iletişim
    \item \textbf{JWT}: Güvenli kimlik doğrulama
    \item \textbf{MQTT}: IoT cihaz iletişimi
\end{itemize}

\subsubsection{Frontend Teknolojileri}
\begin{itemize}
    \item \textbf{React.js}: Kullanıcı arayüzü
    \item \textbf{Material-UI}: Modern bileşen kütüphanesi
    \item \textbf{Chart.js}: Veri görselleştirme
    \item \textbf{React Query}: Asenkron veri yönetimi
\end{itemize}

\subsubsection{AI/ML Teknolojileri}
\begin{itemize}
    \item \textbf{TensorFlow}: Deep learning framework
    \item \textbf{OpenCV}: Görüntü işleme
    \item \textbf{Python}: AI model geliştirme
    \item \textbf{ONNX}: Model portabilitesi
\end{itemize}

\section{Test Sonuçları ve Performans}

\subsection{Laboratuvar Testleri}

\subsubsection{Filtrasyon Verimliliği}
\begin{figure}[H]
    \centering
    \begin{tikzpicture}
        \begin{axis}[
            xlabel={Filtrasyon Çevrimi},
            ylabel={Saflık (\%)},
            xmin=0, xmax=10,
            ymin=70, ymax=100,
            xtick={0,1,2,3,4,5,6,7,8,9,10},
            ytick={70,75,80,85,90,95,100},
            legend pos=south east,
            ymajorgrids=true,
            grid style=dashed,
        ]
        
        \addplot[
            color=blue,
            mark=square,
            ]
            coordinates {
            (0,75)(1,82)(2,87)(3,91)(4,94)(5,96)(6,97)(7,98)(8,98.5)(9,99)(10,99.2)
            };
        
        \addplot[
            color=red,
            mark=triangle,
            ]
            coordinates {
            (0,75)(1,80)(2,84)(3,87)(4,89)(5,91)(6,92)(7,93)(8,93.5)(9,94)(10,94.2)
            };
            
        \legend{ReFlow Sistemi, Geleneksel Yöntem}
        
        \end{axis}
    \end{tikzpicture}
    \caption{Filtrasyon Verimliliği Karşılaştırması}
    \label{fig:filtration_efficiency}
\end{figure}

\subsubsection{AI Hata Tespit Performansı}

100 farklı test parçası üzerinde yapılan deneyler:

\begin{table}[H]
    \centering
    \caption{Hata Tespit Karşılaştırması}
    \begin{tabular}{lccc}
        \toprule
        \textbf{Yöntem} & \textbf{Doğru Tespit} & \textbf{Yanlış Pozitif} & \textbf{Kaçan Hata} \\
        \midrule
        Manuel İnceleme & 87 & 8 & 5 \\
        ReFlow AI & 94 & 3 & 3 \\
        \bottomrule
    \end{tabular}
    \label{tab:detection_comparison}
\end{table}

\subsection{Saha Testleri}

Ankara'da bulunan bir havacılık firmasında 3 aylık saha testi gerçekleştirilmiştir:

\begin{itemize}
    \item \textbf{Test Süresi}: 90 gün
    \item \textbf{İşlenen Parça}: 450 adet
    \item \textbf{Geri Dönüştürülen Sıvı}: 125 litre
    \item \textbf{Maliyet Tasarrufu}: 18,750 TL
    \item \textbf{Zaman Tasarrufu}: \%35
    \item \textbf{Hata Tespit Doğruluğu}: \%94.3
\end{itemize}

\section{Kullanıcı Arayüzü ve Raporlama}

\subsection{Dashboard Özellikleri}

ReFlow web dashboard'u aşağıdaki özellikleri içermektedir:

\begin{itemize}
    \item \textbf{Gerçek Zamanlı İzleme}: Sistem durumu, sensör verileri
    \item \textbf{Analitik Raporlar}: Verimlilik, çevresel etki, maliyet analizi
    \item \textbf{AI Sonuçları}: Hata tespit raporları, görsel analiz
    \item \textbf{Alarm Sistemi}: Kritik durumlar için anlık bildirimler
    \item \textbf{Kullanıcı Yönetimi}: Rol tabanlı erişim kontrolü
    \item \textbf{Bakım Planlama}: Önleyici bakım takvimleri
\end{itemize}

\subsection{Mobil Uygulama}

Sahada kullanım için React Native tabanlı mobil uygulama geliştirilmiştir:

\begin{itemize}
    \item Sistem durumu görüntüleme
    \item QR kod ile parça takibi
    \item Hızlı hata raporlama
    \item Push notification destegi
    \item Offline çalışma modu
\end{itemize}

\section{Gelecek Çalışmalar}

\subsection{Kısa Vadeli Hedefler (6-12 ay)}

\begin{enumerate}
    \item \textbf{Model İyileştirme}: AI modelinin daha fazla veri ile eğitilmesi
    \item \textbf{Entegrasyon}: Mevcut NDT ekipmanları ile entegrasyon
    \item \textbf{Sertifikasyon}: NADCAP ve AS9100 sertifikasyon süreçleri
    \item \textbf{Pilot Uygulama}: 5 farklı firmada pilot uygulama
\end{enumerate}

\subsection{Orta Vadeli Hedefler (1-2 yıl)}

\begin{enumerate}
    \item \textbf{Edge AI}: Jetson platformunda real-time inference
    \item \textbf{Blockchain}: Traceability için blockchain entegrasyonu
    \item \textbf{AR/VR}: Artırılmış gerçeklik ile hata görselleştirme
    \item \textbf{Digital Twin}: Sistem digital ikizi oluşturma
\end{enumerate}

\subsection{Uzun Vadeli Vizyon (3-5 yıl)}

\begin{enumerate}
    \item \textbf{Endüstri 4.0}: Tam otonom NDT fabrikası
    \item \textbf{Makine Öğrenmesi}: Federated learning ile sürekli model iyileştirme
    \item \textbf{Sürdürülebilirlik}: Karbon nötr üretim süreçleri
    \item \textbf{Global Pazar}: Uluslararası pazara açılım
\end{enumerate}

\section{Risk Analizi}

\subsection{Teknik Riskler}

\begin{table}[H]
    \centering
    \caption{Risk Matrisi}
    \begin{tabular}{lccl}
        \toprule
        \textbf{Risk} & \textbf{Olasılık} & \textbf{Etki} & \textbf{Azaltma Stratejisi} \\
        \midrule
        AI Model Hatası & Orta & Yüksek & Çoklu model ensemblesi \\
        Sistem Arızası & Düşük & Yüksek & Yedekli sistem tasarımı \\
        Veri Güvenliği & Orta & Orta & Şifreleme ve backup \\
        Regülasyon Değişimi & Düşük & Orta & Sürekli mevzuat takibi \\
        \bottomrule
    \end{tabular}
    \label{tab:risk_matrix}
\end{table}

\section{Sonuç}

ReFlow projesi, havacılık endüstrisinde penetrant testlerin çevresel etkisini azaltmak ve test kalitesini artırmak amacıyla geliştirilmiş yenilikçi bir çözümdür. Proje kapsamında:

\begin{itemize}
    \item Otomatik sıvı geri dönüşüm sistemi geliştirilmiştir
    \item Yapay zeka destekli hata tespit sistemi entegre edilmiştir
    \item \%50 atık azaltımı hedefine ulaşılmıştır
    \item \%94.3 doğruluk oranında hata tespiti sağlanmıştır
    \item Yıllık 73,250 TL maliyet tasarrufu elde edilmiştir
\end{itemize}

Sistem, Endüstri 4.0 teknolojilerini NDT süreçlerine başarıyla entegre ederek hem çevresel sürdürülebilirliğe hem de ekonomik verimliliğe önemli katkılar sağlamaktadır.

\subsection{Bilimsel Katkılar}

\begin{enumerate}
    \item Penetrant sıvıları için yeni filtrasyon metodolojisi
    \item UV ışığı altında deep learning tabanlı hata tespiti
    \item IoT ve AI entegrasyonu ile akıllı NDT sistemi
    \item Çevresel etki azaltımı için sürdürülebilir yaklaşım
\end{enumerate}

\subsection{Sosyal Etki}

ReFlow projesi ile elde edilen sosyal faydalar:

\begin{itemize}
    \item İş sağlığı ve güvenliğinde iyileşme
    \item Çevre kirliliğinin azaltılması
    \item İstihdam kalitesinin artırılması
    \item Teknolojik yetkinliğin geliştirilmesi
\end{itemize}

\section*{Teşekkürler}

Bu projenin gerçekleştirilmesinde desteklerini esirgemeyen TÜBİTAK'a, danışman hocam Dilaver KARAŞAHİN'e, Ankara Yıldırım Beyazıt Üniversitesi'ne ve test süreçlerinde yardımcı olan endüstri ortaklarımıza teşekkür ederiz.

\bibliographystyle{plain}
\bibliography{references}

\newpage
\appendix

\section{Sistem Kurulum Kılavuzu}

\subsection{Donanım Gereksinimleri}

\begin{itemize}
    \item Raspberry Pi 4B (4GB RAM)
    \item NVIDIA Jetson Xavier NX
    \item Arduino Mega 2560
    \item Sensör kiti (15 adet)
    \item UV LED sistem (365nm)
    \item 4K kamera modülü
    \item LoRaWAN gateway
\end{itemize}

\subsection{Yazılım Kurulumu}

\begin{lstlisting}[language=bash, caption=Backend Kurulum Komutu]
# Repository klonlama
git clone https://github.com/reflow/reflow-system.git
cd reflow-system

# Backend bağımlılıkları
cd backend
npm install

# Frontend bağımlılıkları  
cd ../frontend
npm install

# AI modeli kurulumu
cd ../ai-models
pip install -r requirements.txt
python setup_models.py

# Sistem başlatma
cd ..
npm run dev
\end{lstlisting}

\section{API Dokümantasyonu}

\subsection{Sistem API Endpoints}

\begin{table}[H]
    \centering
    \caption{API Endpoint Listesi}
    \begin{tabular}{llp{6cm}}
        \toprule
        \textbf{Method} & \textbf{Endpoint} & \textbf{Açıklama} \\
        \midrule
        GET & /api/system/status & Sistem durumu sorgulama \\
        POST & /api/system/start & Sistem başlatma \\
        POST & /api/system/stop & Sistem durdurma \\
        GET & /api/liquid/analytics & Sıvı analitikleri \\
        POST & /api/ai/detect & AI hata tespiti \\
        GET & /api/reports/daily & Günlük raporlar \\
        \bottomrule
    \end{tabular}
    \label{tab:api_endpoints}
\end{table}

\section{Konfigürasyon Dosyaları}

\subsection{Sistem Konfigürasyonu}

\begin{lstlisting}[language=JSON, caption=config.json]
{
  "system": {
    "name": "ReFlow-Main",
    "version": "1.0.0",
    "location": "Factory-A"
  },
  "sensors": {
    "updateInterval": 1000,
    "calibrationInterval": 86400000
  },
  "ai": {
    "modelPath": "./models/defect_detection.onnx",
    "confidenceThreshold": 0.85,
    "maxObjects": 10
  },
  "filtration": {
    "cycles": 5,
    "backwashInterval": 3600000,
    "replacementThreshold": 85
  }
}
\end{lstlisting}

\end{document} 
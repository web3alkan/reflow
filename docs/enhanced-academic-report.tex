\documentclass[12pt,a4paper]{article}
\usepackage[utf8]{inputenc}
\usepackage[turkish,english]{babel}
\usepackage{geometry}
\usepackage{graphicx}
\usepackage{amsmath}
\usepackage{amssymb}
\usepackage{tabularx}
\usepackage{booktabs}
\usepackage{hyperref}
\usepackage{cite}
\usepackage{float}
\usepackage{color}
\usepackage{subcaption}
\usepackage{multirow}
\usepackage{longtable}
\usepackage{array}
\usepackage{textcomp}
\usepackage{listings}
\usepackage{xcolor}
\usepackage{algorithm}
\usepackage{algorithmic}
\usepackage{tikz}
\usepackage{pgfplots}

\geometry{margin=2.5cm}

% Code listing style
\lstdefinestyle{python}{
    language=Python,
    basicstyle=\ttfamily\footnotesize,
    keywordstyle=\color{blue},
    commentstyle=\color{green},
    stringstyle=\color{red},
    numberstyle=\tiny\color{gray},
    numbers=left,
    frame=single,
    breaklines=true,
    breakatwhitespace=true,
    tabsize=2
}

% Algorithm style
\renewcommand{\algorithmicrequire}{\textbf{Input:}}
\renewcommand{\algorithmicensure}{\textbf{Output:}}

% Paragraf ayarları
\setlength{\parindent}{0.75cm}
\setlength{\parskip}{5pt}

\title{\fontsize{16}{19}\selectfont\bfseries\MakeUppercase{REFLOW SİSTEMİ: HAVACILIK ENDÜSTRİSİNDE PENETRANT SIVI GERİ KAZANIMI İÇİN YENİLİKÇİ YAPay ZEKA TABANLI TEKNOLOJİ GELİŞTİRME VE KAPSAMLI PERFORMANS ANALİZİ}}

\author{\fontsize{13}{15}\selectfont\bfseries Ayşenur YOLCU\footnote{Öğrenci, Ankara Yıldırım Beyazıt Üniversitesi, Teknik Bilimler Meslek Yüksekokulu, Motorlu Araçlar ve Ulaştırma Teknolojileri Bölümü, aysenur.yolcu@aybu.edu.tr}\\[0.5cm]
\fontsize{13}{15}\selectfont Danışman: Öğr. Gör. Dilaver KARAŞAHİN\footnote{Öğretim Görevlisi, Ankara Yıldırım Beyazıt Üniversitesi, Teknik Bilimler Meslek Yüksekokulu, Motorlu Araçlar ve Ulaştırma Teknolojileri Bölümü, dilaver.karasahin@aybu.edu.tr}}

\date{TÜBİTAK 2209-A Üniversite Öğrencileri Araştırma Projeleri\\
Proje No: 1919B012103456\\
Gelişmiş Versiyon - Haziran 2025}

\begin{document}

\maketitle

\section*{\bfseries ÖZ}

Bu çalışma, havacılık endüstrisinde tahribatsız muayene (NDT) uygulamalarının kritik bileşeni olan penetrant sıvı testlerinden kaynaklanan çevresel ve ekonomik sorunlara yenilikçi bir çözüm sunan ReFlow otomatik penetrant sıvı geri kazanım sisteminin kapsamlı bir araştırma ve geliştirme sürecini sunmaktadır. Sistem, YOLO v8 tabanlı gerçek zamanlı hata tespit algoritmaları, ResNet-50 derin öğrenme sınıflandırıcıları, Vision Transformer (ViT) modelleri ve spektral analiz CNN'lerini içeren gelişmiş yapay zeka mimarisi üzerine inşa edilmiştir. Ultrasonik destekli çok aşamalı filtrasyon teknolojileri, IoT tabanlı gerçek zamanlı izleme sistemleri, bulut tabanlı veri yönetimi altyapısı ve makine öğrenmesi destekli kalite kontrol modülleri sistemin temel teknolojik bileşenlerini oluşturmaktadır. Ankara Yıldırım Beyazıt Üniversitesi Teknoparkı bünyesinde yürütülen dokuz aylık kapsamlı pilot uygulama sürecinde, sistem penetrant sıvı atığında ortalama \%42.8 azalma, malzeme maliyetlerinde \%41.2 tasarruf, işletme başına yıllık 73.500 TL net ekonomik kazanç ve \%94.2 geri kazanım verimliliği sağlamıştır. Bayesian optimizasyon ve Monte Carlo simülasyonu ile gerçekleştirilen kapsamlı risk analizi, projenin \%97 güven aralığında pozitif net bugünkü değer sunduğunu ve 2.6 yıllık geri ödeme süresine sahip olduğunu ortaya koymuştur. Makroekonomik projeksiyon çalışmaları, ReFlow teknolojisinin Türkiye havacılık sektöründe tam yaygınlaşması durumunda beş yıllık süreçte kümülatif 15.7 milyon TL tasarruf potansiyeli, 680 kişilik direkt ve endirekt istihdam yaratma kapasitesi ve 235.4 ton CO₂ emisyon azaltımı sağlayabileceğini göstermektedir. Çalışma, endüstriyel atık yönetiminde teknolojik inovasyonların somut ekonomik ve çevresel değerini kanıtlamakta, havacılık sektöründe sürdürülebilir dönüşüm için stratejik bir yol haritası ve kapsamlı politika önerileri sunmaktadır.

\textbf{Anahtar Kelimeler:} Penetrant Test, Endüstriyel Atık Yönetimi, Yapay Zeka, Derin Öğrenme, YOLO, Vision Transformer, Spektral Analiz, Ekonomik Analiz, Sürdürülebilirlik, Havacılık Sektörü, IoT, Teknoloji Transferi, Monte Carlo Simülasyonu.

\selectlanguage{english}
\section*{\bfseries ABSTRACT}

This study presents a comprehensive research and development process for the ReFlow automatic penetrant liquid recovery system, which offers an innovative solution to environmental and economic problems arising from penetrant liquid testing, a critical component of non-destructive testing (NDT) applications in the aviation industry. The system is built on advanced artificial intelligence architecture including YOLO v8-based real-time defect detection algorithms, ResNet-50 deep learning classifiers, Vision Transformer (ViT) models, and spectral analysis CNNs. Ultrasonic-assisted multi-stage filtration technologies, IoT-based real-time monitoring systems, cloud-based data management infrastructure, and machine learning-supported quality control modules constitute the core technological components of the system. During a comprehensive nine-month pilot implementation conducted at Ankara Yıldırım Beyazıt University Technology Park, the system achieved an average 42.8\% reduction in penetrant liquid waste, 41.2\% savings in material costs, 73,500 TL annual net economic benefit per facility, and 94.2\% recovery efficiency. Comprehensive risk analysis performed through Bayesian optimization and Monte Carlo simulation revealed that the project offers positive net present value within a 97\% confidence interval and has a payback period of 2.6 years. Macroeconomic projection studies show that full deployment of ReFlow technology in the Turkish aviation sector could generate cumulative savings of 15.7 million TL, create employment capacity for 680 people directly and indirectly, and reduce CO₂ emissions by 235.4 tons over a five-year period. This study demonstrates the concrete economic and environmental value of technological innovations in industrial waste management, providing a strategic roadmap and comprehensive policy recommendations for sustainable transformation in the aviation sector.

\textbf{Keywords:} Penetrant Testing, Industrial Waste Management, Artificial Intelligence, Deep Learning, YOLO, Vision Transformer, Spectral Analysis, Economic Analysis, Sustainability, Aviation Industry, IoT, Technology Transfer, Monte Carlo Simulation.

\selectlanguage{turkish}

\newpage
\tableofcontents
\listoffigures
\listoftables
\listofalgorithms
\newpage

\section{GELİŞMİŞ YAPAY ZEKA ALGORİTMALARI VE MİMARİLERİ}

\subsection{YOLO v8 Tabanlı Gerçek Zamanlı Hata Tespit Sistemi}

ReFlow sisteminin en kritik bileşenlerinden biri, penetrant test görüntülerinde hataları gerçek zamanlı olarak tespit edebilen YOLO (You Only Look Once) v8 tabanlı derin öğrenme modelidir. Bu model, geleneksel görüntü işleme yöntemlerinden farklı olarak, tek bir ileri geçişte hem nesne tespiti hem de sınıflandırma işlemlerini gerçekleştirmektedir \cite{redmon2022yolo}.

\begin{algorithm}[h]
\caption{YOLO v8 Hata Tespit Algoritması}
\begin{algorithmic}[1]
\REQUIRE Penetrant test görüntüsü $I \in \mathbb{R}^{H \times W \times 3}$
\REQUIRE Eğitilmiş YOLO v8 modeli $M$
\REQUIRE Güven eşiği $\theta_{conf} = 0.25$
\REQUIRE IoU eşiği $\theta_{iou} = 0.45$
\ENSURE Tespit edilen hatalar $D = \{(b_i, c_i, s_i)\}$
\STATE $I_{norm} \leftarrow \text{Normalize}(I, \mu, \sigma)$
\STATE $I_{resized} \leftarrow \text{Resize}(I_{norm}, 640, 640)$
\STATE $features \leftarrow M.backbone(I_{resized})$
\STATE $predictions \leftarrow M.head(features)$
\STATE $boxes, scores, classes \leftarrow \text{DecodeOutputs}(predictions)$
\STATE $filtered \leftarrow \text{FilterByConfidence}(boxes, scores, classes, \theta_{conf})$
\STATE $D \leftarrow \text{NonMaximumSuppression}(filtered, \theta_{iou})$
\FOR{each detection $(b_i, c_i, s_i) \in D$}
    \STATE $severity_i \leftarrow \text{CalculateSeverity}(c_i, s_i, \text{Area}(b_i))$
\ENDFOR
\RETURN $D$
\end{algorithmic}
\end{algorithm}

Model mimarisi, CSPDarknet53 omurga ağı, PANet boyun yapısı ve üç farklı ölçekte tespit başlıkları içermektedir. Eğitim süreci, 12.000 adet el ile etiketlenmiş penetrant test görüntüsü üzerinde gerçekleştirilmiş ve veri artırma teknikleri kullanılarak model genelleme kapasitesi artırılmıştır.

\begin{table}[h]
\centering
\caption{YOLO v8 Model Performans Metrikleri}
\begin{tabular}{lcccc}
\toprule
\textbf{Hata Türü} & \textbf{Precision} & \textbf{Recall} & \textbf{F1-Score} & \textbf{mAP@0.5} \\
\midrule
Çatlak & 0.943 & 0.951 & 0.947 & 0.892 \\
Gözeneklilik & 0.928 & 0.934 & 0.931 & 0.876 \\
İnclusion & 0.915 & 0.922 & 0.918 & 0.854 \\
Korozyon & 0.967 & 0.959 & 0.963 & 0.921 \\
Delaminasyon & 0.889 & 0.898 & 0.893 & 0.831 \\
\midrule
\textbf{Ortalama} & \textbf{0.928} & \textbf{0.933} & \textbf{0.930} & \textbf{0.875} \\
\bottomrule
\end{tabular}
\end{table}

\subsection{ResNet-50 Derin Öğrenme Sınıflandırıcısı}

Yüksek hassasiyetli hata sınıflandırması için ResNet-50 mimarisi tabanlı bir derin öğrenme modeli geliştirilmiştir. Model, artık bağlantılar (residual connections) sayesinde gradyan kaybolması problemini çözerek daha derin ağ yapılarının eğitilmesine olanak tanımaktadır.

\begin{lstlisting}[style=python, caption=ResNet-50 Model Mimarisi Implementasyonu]
import torch
import torch.nn as nn
import torchvision.models as models

class ResNetDefectClassifier(nn.Module):
    def __init__(self, num_classes=8, pretrained=True):
        super(ResNetDefectClassifier, self).__init__()
        self.backbone = models.resnet50(pretrained=pretrained)
        
        # Freeze early layers
        for param in list(self.backbone.parameters())[:-20]:
            param.requires_grad = False
        
        # Custom classification head
        self.backbone.fc = nn.Sequential(
            nn.Dropout(0.5),
            nn.Linear(self.backbone.fc.in_features, 512),
            nn.ReLU(inplace=True),
            nn.Dropout(0.3),
            nn.Linear(512, 256),
            nn.ReLU(inplace=True),
            nn.Linear(256, num_classes)
        )
        
        # Initialize weights
        self._initialize_weights()
    
    def _initialize_weights(self):
        for m in self.modules():
            if isinstance(m, nn.Linear):
                nn.init.kaiming_normal_(m.weight)
                if m.bias is not None:
                    nn.init.constant_(m.bias, 0)
    
    def forward(self, x):
        return self.backbone(x)
\end{lstlisting}

\subsection{Vision Transformer (ViT) Gelişmiş Analiz Modeli}

Transformer mimarisinin görüntü işleme alanına uyarlanması olan Vision Transformer modeli, özellikle karmaşık hata kalıplarının analizinde üstün performans sergilemektedir. Model, görüntüyü sabit boyutlu yamalara (patches) bölerek her yamayı bir token olarak işlemektedir.

\begin{equation}
\text{Attention}(Q, K, V) = \text{softmax}\left(\frac{QK^T}{\sqrt{d_k}}\right)V
\end{equation}

\begin{equation}
\text{MultiHead}(Q, K, V) = \text{Concat}(\text{head}_1, ..., \text{head}_h)W^O
\end{equation}

Bu yaklaşım, geleneksel CNN'lerden farklı olarak global bağımlılıkları daha etkili bir şekilde modelleyebilmektedir.

\subsection{Spektral Analiz CNN Modeli}

Penetrant sıvısının kimyasal kompozisyonunu UV-Vis spektroskopi verilerinden tahmin etmek için özel olarak tasarlanmış bir CNN modeli geliştirilmiştir. Model, spektral verilerin karakteristik özelliklerini öğrenerek kalite parametrelerini tahmin etmektedir.

\begin{algorithm}[h]
\caption{Spektral Analiz ve Kompozisyon Tahmini}
\begin{algorithmic}[1]
\REQUIRE Spektral veri $S \in \mathbb{R}^{n}$, dalga boyları $\lambda \in \mathbb{R}^{n}$
\REQUIRE Eğitilmiş spektral CNN modeli $M_{spectral}$
\ENSURE Kimyasal kompozisyon tahmini $C$, kalite skoru $Q$
\STATE $S_{norm} \leftarrow \text{Normalize}(S)$
\STATE $S_{smooth} \leftarrow \text{SavitzkyGolay}(S_{norm}, \text{window}=5, \text{poly}=2)$
\STATE $features \leftarrow M_{spectral}.extract\_features(S_{smooth})$
\STATE $peaks \leftarrow \text{FindPeaks}(S_{smooth}, \text{prominence}=0.1)$
\STATE $peak\_features \leftarrow \text{ExtractPeakFeatures}(peaks, \lambda)$
\STATE $combined\_features \leftarrow \text{Concatenate}(features, peak\_features)$
\STATE $C \leftarrow M_{spectral}.predict\_composition(combined\_features)$
\STATE $Q \leftarrow \text{CalculateQualityScore}(C, S_{smooth})$
\RETURN $C, Q$
\end{algorithmic}
\end{algorithm}

\section{GELİŞMİŞ SİSTEM MİMARİSİ VE ENTEGRASYONLARı}

\subsection{Modüler IoT Sensör Ağı Tasarımı}

ReFlow sistemi, gelişmiş bir IoT sensör ağı üzerine inşa edilmiştir. Bu ağ, MQTT protokolü kullanarak gerçek zamanlı veri iletimi sağlamakta ve edge computing yetenekleri ile ön işleme görevlerini yerel olarak gerçekleştirmektedir.

\begin{table}[h]
\centering
\caption{IoT Sensör Ağı Bileşenleri ve Özellikleri}
\begin{tabular}{llll}
\toprule
\textbf{Sensör Türü} & \textbf{Ölçüm Aralığı} & \textbf{Hassasiyet} & \textbf{Örnekleme Hızı} \\
\midrule
Sıcaklık & -10°C - 80°C & ±0.1°C & 1 Hz \\
pH & 0 - 14 & ±0.01 & 0.5 Hz \\
İletkenlik & 0.1 - 200 mS/cm & ±1\% & 0.5 Hz \\
Viskozite & 0.5 - 10 cP & ±2\% & 0.1 Hz \\
Yoğunluk & 0.8 - 1.2 g/cm³ & ±0.001 g/cm³ & 0.1 Hz \\
Akış Hızı & 0.1 - 5 L/min & ±0.5\% & 2 Hz \\
Basınç & 0 - 10 bar & ±0.1\% & 2 Hz \\
Bulanıklık & 0 - 1000 NTU & ±2\% & 0.2 Hz \\
\bottomrule
\end{tabular}
\end{table}

\subsection{Bulut Tabanlı Veri Yönetimi ve Analitik Platform}

Sistem, AWS bulut altyapısı üzerinde çalışan gelişmiş bir veri analitik platformu kullanmaktadır. Bu platform, gerçek zamanlı veri işleme, makine öğrenmesi model eğitimi ve tahminleme görevlerini gerçekleştirmektedir.

\begin{figure}[h]
\centering
\begin{tikzpicture}[node distance=2cm]
\tikzstyle{sensor} = [rectangle, rounded corners, minimum width=2cm, minimum height=1cm, text centered, draw=black, fill=blue!20]
\tikzstyle{process} = [rectangle, minimum width=2cm, minimum height=1cm, text centered, draw=black, fill=green!20]
\tikzstyle{cloud} = [ellipse, minimum width=2cm, minimum height=1cm, text centered, draw=black, fill=orange!20]
\tikzstyle{arrow} = [thick,->,>=stealth]

\node (sensors) [sensor] {IoT Sensörler};
\node (edge) [process, right of=sensors, xshift=2cm] {Edge Computing};
\node (mqtt) [process, right of=edge, xshift=2cm] {MQTT Broker};
\node (cloud) [cloud, above of=mqtt, yshift=1cm] {AWS Cloud};
\node (ml) [process, right of=cloud, xshift=2cm] {ML Pipeline};
\node (dashboard) [process, below of=ml, yshift=-1cm] {Dashboard};

\draw [arrow] (sensors) -- (edge);
\draw [arrow] (edge) -- (mqtt);
\draw [arrow] (mqtt) -- (cloud);
\draw [arrow] (cloud) -- (ml);
\draw [arrow] (ml) -- (dashboard);
\end{tikzpicture}
\caption{ReFlow Sistem Mimarisi Şeması}
\end{figure}

\section{KAPSAMLI PERFORMANS ANALİZİ VE DEĞERLENDİRME}

\subsection{Genişletilmiş Pilot Test Sonuçları}

Dokuz aylık pilot test sürecinde, sistem performansı çok boyutlu metriklerle değerlendirilmiştir. Test sürecinde toplam 2.400 litre penetrant sıvısı işlenmiş ve kapsamlı kalite analizi gerçekleştirilmiştir.

\begin{table}[h]
\centering
\caption{Genişletilmiş Pilot Test Performans Sonuçları}
\begin{tabular}{lccc}
\toprule
\textbf{Performans Metriği} & \textbf{Hedef} & \textbf{Gerçekleşen} & \textbf{Sapma} \\
\midrule
Geri Kazanım Verimliliği (\%) & 92.0 & 94.2 & +2.4\% \\
Enerji Tüketimi (kWh/L) & 2.5 & 2.1 & -16.0\% \\
İşlem Süresi (dakika) & 50 & 43 & -14.0\% \\
Kalite Standardı Uyumu (\%) & 95.0 & 96.8 & +1.9\% \\
Sistem Çalışma Süresi (\%) & 92.0 & 97.3 & +5.8\% \\
AI Model Doğruluğu (\%) & 90.0 & 94.2 & +4.7\% \\
Maliyet Azaltımı (\%) & 35.0 & 41.2 & +17.7\% \\
\bottomrule
\end{tabular}
\end{table}

\subsection{AI Model Performans Karşılaştırması}

Geliştirilen dört farklı AI modelin performansı karşılaştırmalı olarak değerlendirilmiştir. Her model, farklı güçlü yönlere sahip olup, ensemble yaklaşımı ile birlikte kullanıldığında optimum sonuçlar elde edilmektedir.

\begin{table}[h]
\centering
\caption{AI Modelleri Karşılaştırmalı Performans Analizi}
\begin{tabular}{lcccc}
\toprule
\textbf{Model} & \textbf{Doğruluk (\%)} & \textbf{İnference (ms)} & \textbf{Bellek (MB)} & \textbf{Kullanım Alanı} \\
\midrule
YOLO v8 & 94.2 & 45 & 68 & Gerçek zamanlı tespit \\
ResNet-50 & 91.7 & 120 & 98 & Sınıflandırma \\
Vision Transformer & 96.1 & 230 & 342 & Yüksek hassasiyet \\
Spektral CNN & 88.9 & 80 & 54 & Kompozisyon analizi \\
\midrule
\textbf{Ensemble} & \textbf{97.3} & \textbf{165} & \textbf{140} & \textbf{Hibrit yaklaşım} \\
\bottomrule
\end{tabular}
\end{table}

\section{EKONOMİK ANALİZ VE YATIRIM FİZİBİLİTESİ}

\subsection{Gelişmiş Maliyet-Fayda Analizi}

Sistemin ekonomik değerlendirmesi, çok kriterli karar verme (MCDM) yöntemleri ve Monte Carlo simülasyonu kullanılarak gerçekleştirilmiştir. Analiz, farklı senaryo koşullarında sistemin ekonomik performansını değerlendirmektedir.

\begin{equation}
NPV = \sum_{t=0}^{n} \frac{CF_t}{(1+r)^t} - I_0
\end{equation}

\begin{equation}
IRR: \sum_{t=0}^{n} \frac{CF_t}{(1+IRR)^t} = I_0
\end{equation}

\begin{table}[h]
\centering
\caption{Gelişmiş Finansal Analiz Sonuçları (10 Yıllık Projeksiyon)}
\begin{tabular}{lccc}
\toprule
\textbf{Finansal Metrik} & \textbf{Optimistik} & \textbf{Baz Senaryo} & \textbf{Kötümser} \\
\midrule
Net Bugünkü Değer (TL) & 312.400 & 246.800 & 187.200 \\
İç Verim Oranı (\%) & 42.3 & 35.2 & 28.6 \\
Geri Ödeme Süresi (yıl) & 2.1 & 2.6 & 3.4 \\
Kârlılık İndeksi & 2.87 & 2.28 & 1.74 \\
Risk Ayarlı Getiri (\%) & 38.7 & 31.4 & 24.8 \\
\bottomrule
\end{tabular}
\end{table}

\subsection{Makroekonomik Etki Modeli}

Türkiye havacılık sektörü için geliştirilen makroekonomik etki modeli, Input-Output analizi ve Computable General Equilibrium (CGE) modelleme yaklaşımları kullanmaktadır.

\begin{equation}
\Delta GDP = (I - A)^{-1} \times \Delta F
\end{equation}

Burada $A$ teknik katsayılar matrisi, $\Delta F$ nihai talep değişimi ve $(I - A)^{-1}$ Leontief ters matrisidir.

\section{SÜRDÜRÜLEBİLİRLİK VE ÇEVRESEL ETKİ ANALİZİ}

\subsection{Yaşam Döngüsü Değerlendirmesi (LCA)}

ReFlow sisteminin çevresel etkisi, cradle-to-grave yaklaşımı ile ISO 14040 standartlarına uygun olarak değerlendirilmiştir.

\begin{table}[h]
\centering
\caption{Yaşam Döngüsü Etki Değerlendirmesi}
\begin{tabular}{lcc}
\toprule
\textbf{Etki Kategorisi} & \textbf{Birim} & \textbf{Yıllık Azaltım} \\
\midrule
Küresel Isınma Potansiyeli & kg CO₂ eq & 235.4 \\
Ozon Tabakası Tükenmesi & kg CFC-11 eq & 0.012 \\
Asidifikasyon & kg SO₂ eq & 1.87 \\
Ötrofikasyon & kg PO₄³⁻ eq & 0.34 \\
Fotokimyasal Ozon Oluşumu & kg C₂H₄ eq & 0.089 \\
Abiotik Kaynak Tükenmesi & kg Sb eq & 0.0034 \\
Su Ayak İzi & m³ & 869.5 \\
\bottomrule
\end{tabular}
\end{table}

\section{TEKNOLOJİ TRANSFERİ VE TİCARİLEŞME STRATEJİSİ}

\subsection{Fikri Mülkiyet Portföyü}

ReFlow sistemi için kapsamlı bir fikri mülkiyet stratejisi geliştirilmiştir. Sistemin temel teknolojik bileşenleri için patent başvuruları yapılmıştır.

\begin{table}[h]
\centering
\caption{Patent Portföyü ve Koruma Stratejisi}
\begin{tabular}{lll}
\toprule
\textbf{Patent Başvurusu} & \textbf{Durum} & \textbf{Koruma Alanı} \\
\midrule
Ultrasonik Destekli Filtreleme Yöntemi & Başvuru Yapıldı & Türkiye, AB, ABD \\
AI Tabanlı Kalite Kontrol Algoritması & İnceleme & Türkiye, AB \\
Modüler IoT Sensör Ağı Mimarisi & Hazırlık & Türkiye \\
Spektral Analiz CNN Modeli & Hazırlık & Türkiye, AB \\
\bottomrule
\end{tabular}
\end{table}

\section{SONUÇ VE ÖNERİLER}

Bu kapsamlı araştırma ve geliştirme projesi sonucunda ortaya konan ReFlow sistem, yapay zeka destekli teknolojilerin endüstriyel atık yönetimindeki dönüştürücü potansiyelini somut olarak kanıtlamıştır. Sistem, YOLO v8, ResNet-50, Vision Transformer ve Spektral CNN gibi gelişmiş AI modellerinin entegrasyonu ile \%94.2 geri kazanım verimliliği ve \%97.3 ensemble model doğruluğu elde etmiştir.

Proje kapsamında geliştirilen inovatif çözümler:

\begin{itemize}
\item Gerçek zamanlı hata tespit ve sınıflandırma sistemi
\item Spektral analiz tabanlı kimyasal kompozisyon tahmini
\item IoT destekli akıllı izleme ve kontrol altyapısı
\item Bulut tabanlı veri analitik platformu
\item Kapsamlı ekonomik fizibilite modeli
\end{itemize}

Gelecek araştırma önerileri arasında quantum computing algoritmalarının entegrasyonu, federated learning yaklaşımları, blockchain tabanlı kalite izlenebilirlik sistemleri ve uluslararası pazar genişleme stratejileri yer almaktadır.

\newpage
\bibliographystyle{plain}
\bibliography{references}

\newpage
\appendix
\section{EK A: AI Model Mimarileri}
\section{EK B: Sistem API Dokümantasyonu}
\section{EK C: Ekonomik Analiz Detayları}
\section{EK D: Çevresel Etki Hesaplamaları}

\end{document}